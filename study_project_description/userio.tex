SŦYX has built-in functions for user input and output that work with all supported data types. 
\textcolor{teal}{ẞ©Æ’(}variable\textcolor{teal}{)} reads a value 
from the standard input and assigns it to the variable.
\textcolor{teal}{Þ®ı’Ŧ(}variable\textcolor{teal}{)} prints the value of the variable to the standard output.
There also is the possibility to call the print function with an additional width parameter
like this: \textcolor{teal}{Þ®ı’Ŧ(}width\textcolor{teal}{?}variable\textcolor{teal}{)}
This results in the output being formatted to the specified width.\newline
\newline
Random integers beween 0 and a specified value can be generated using the 
\textcolor{teal}{®Æ’Ðı’Ŧ(}variable\_max\_number\textcolor{teal}{)} function,
which returns the generated number.\newline
\newline
Another built-in function is \textcolor{teal}{ẞ›ẞŦ€º(}variable\_string\textcolor{teal}{)},
which allows the user to call a system command  with a string as parameter.
The return value of this function is 0 if the command was executed successfully and non zero otherwise.