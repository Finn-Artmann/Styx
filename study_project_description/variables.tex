Variables can be of the data types described earlier. 
The variable has to be declared before it can be used.
Declarations can be done anywhere in a function body or in the main function.
A variable can also be declared as global variable using the \textcolor{teal}{ŊŁØ‘ÆŁ} keyword,
which makes it accessible from anywhere in the program.
It is possible to directly assign a value to a variable on declaration.
Local variables can not be accessed from outside the function or the scope they are declared in.
Scopes can be created using \textcolor{teal}{°} and \textcolor{teal}{°°} in any
function body or the main function.
\newline
\newline
Examples Declarations:
\begin{itemize}[noitemsep]
    \item Standard declaration: \textcolor{teal}{ÐØ↑‘Ł€\#‚Æ®;}
    \item Declaration with assignment: \textcolor{teal}{ÐØ↑‘Ł€\#‚Æ®\#§\#3.14;}
\end{itemize}

