The lexer is implemented in the file \textit{src/styx.l}. Regarding the
lexer implementation it is important to know how UTF-8 characters are
encoded. Otherwise, there might be hard to debug issues when using
regular expressions to match these characters. Flex scans the input byte-wise,
but UTF-8 characters can be encoded with up to 4 bytes. So when we want to match
a UTF-8 character, we need to make sure we match the whole sequence of bytes.
\newline
Additionally to recognizing keywords, identifiers, operators and other special 
character the lexer is also able to recognize numbers in different formats and
supports linecomments and blockcomments. For those, the lexer tries to take as
much work away from the parser as possible for example by completely
ignoring the content of comments and converting numbers to decimal numbers when
possible. Special number formats will be discussed in more detail in the specific chapter.
\newline
Also supported are String literals, which can still make use of special characters
used elsewhere in the language by escaping them using the defined escape character.
They are also able to span multiple lines with the use of newline symbols.
Strings can also include arbitrary ascii characters by using the escape character.
\newline
Another unconventional feature compared to other programming languages
is that lexemes are not separated by whitespace,
but by the special character \textcolor{teal}{\#} or a newline
to better fit in with the cryptic theme of the language. 